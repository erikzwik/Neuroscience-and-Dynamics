\documentclass[]{article}

%opening
\title{Analysis of studies}
\author{Erik Wik}

\begin{document}

\maketitle


\section{Week 1}

I have barely been able to do anything this first week. It's been hard to find any time to do any studies. Therefore, because I do find it very important to study, and to see where this path can lead me, I will start doing one hour in the morning. This will change my morning routine to the following: meditate, write, study. It would be good to wake up at 6, in order to manage this, but it would be even better if I can do it more naturally. 

Furthermore, I think I'll make the pace half as fast as before. This will make the whole endeavor take roughly 26 weeks, or half a year. Which frightens me, but it's also a much more sensible pace, considering the length I've gotten my first week. 



\subsection{ Gerstner }

Leaky-integrate and fire model of the neuron, as well as it's triad of representation (neuron, circuit and mathematical formula). Solved some simple first-order linear equations which represent the behavior of the neuron. 
$\tau \frac{d}{dt}u = -(u - u_{rest}) + RI(t)$


\subsection{Strogatz }


Quick look at pitchfork bifurcation, both subcritical and supercritical represented by 

$\dot{x} = x + rx^{3}$
$\dot{x} = x - rx^{3}$ (I have to return to what this means later)


where r is the control parameter. Clearly these are the same in $-\infty < r <\infty$. 
So.. I'll look at it again.


One-linear dynamics on a circle. Introduces periodicity, otherwise very simple. 
$\dot{x}$ = x is not a vector field on this topology, because $x|_{2\pi} = x|_{0}$, but $\dot{x}|_{2\pi} \neq \dot{x}_|{0}$. 

\newpage
\section{Week 2}

\end{document}
